%!TeX TXS-program:bibliography = txs:///biber

\documentclass[10pt,letterpaper]{article}
\usepackage[latin1]{inputenc}
\usepackage{amsmath}
\usepackage{amsfonts}
\usepackage{amssymb}
\usepackage{graphicx}
\usepackage{xcolor}
\usepackage{listings}

\usepackage[style=apa,backend=biber]{biblatex}
\addbibresource{https://raw.githubusercontent.com/ncrncornell/cms-to-bib/master/cms.bib}
\title{Sample data citation}
\author{Future Nobel Prize}

\begin{document}
\maketitle
We will use the Annual Survey of Manufactures {\color{red} \parencite{cms1}} to do something.

\section{Details}
This sample \LaTeX document uses \texttt{biblatex} to pull down the \texttt{cms.bib} file. The key code is

\begin{lstlisting}[language=TeX]
\usepackage[style=apa,backend=biber]{biblatex}
\addbibresource{https://raw.githubusercontent.com/ncrncornell/
  cms-to-bib/master/cms.bib}
\end{lstlisting}
This will download the file from a website.

\section{Caveats}
\subsection{Might not work on Overleaf}
There are reports that this code will not work on Overleaf, due to the remote-file download functionality. The workaround is to use the URL-upload facility on Overleaf to add the \texttt{cms.bib} to the Overleaf folder, and change the \texttt{addbibresource} line to read
\begin{lstlisting}
\addbibresource{cms.bib}
\end{lstlisting}

\subsection{Year reported says "release year"}
The supplied bib file is just an example. In practice, it needs to be adapted either by the agency providing it or by the researcher using it. 

\printbibliography	
\end{document}